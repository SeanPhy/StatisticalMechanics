\documentclass{article}

\usepackage[left=1.5cm, right=1.5cm, top=3cm, bottom = 3cm]{geometry}

\usepackage{amsmath}
\usepackage{amsfonts}
\usepackage{amssymb}
\usepackage{graphicx}
\usepackage{float}
\usepackage{wrapfig}
\usepackage{latexsym}
\usepackage{hyperref}
\usepackage{feynmf}
\usepackage{exscale}
\usepackage{relsize}
\usepackage{bm}%bold math, for vector
\linespread{1.1}

%%%%%%%
%第⑦章习题安排:
%%%%%%%
%宋志坚 1,2,3
%宋盛雨央 11,21,31
%陈博文 12,22,32
%解放 13,23,33
%辜晨曦 4,14,24,34
%鲍亦澄 5,15,25,35
%蒋文韬 6,16,26,36
%李嘉琛 7,17,27,37
%颜公望 8,18,28,38
%张传坤 9,19,29
%王志凌 10,20,30

\author{SM-at-THU}
\title{\bf{Solutions to Pathria's Statistical Mechanics}\\Chapter 7}

\begin{document}
\maketitle
\section*{Problem 7.1}

\section*{Problem 7.4}
We can deduce from 7.1.30 and 7.1.26a,when p is a const:
$$g_{\frac{5}{2}}\propto T^{-\frac{5}{2}}$$
So we have:
$$(\frac{\partial}{\partial T}{g_{\frac{5}{2}}(z)})_p=-\frac{5}{2T}  g_{\frac{5}{2}}(z)$$
According to the D.10 from appendix D:
$$z\frac{\partial}{\partial z}g_{\frac{5}{2}}(z)=g_{\frac{3}{2}}(z)$$
Combine the two equation:
$$\frac{1}{z}(\frac{\partial z}{\partial T})_p=-\frac{5}{2T}\frac{g_{\frac{5}{2}}(z)}{g_{\frac{3}{2}}(z)}$$
If you compare with 7.1.36 you can get:
$$\gamma=\frac{5}{3}\frac{g_{\frac{5}{2}}(z)g_{\frac{1}{2}}(z)}{g_{\frac{3}{2}}(z)^2}$$

\section*{Problem 7.11}
\subsection*{(a)}
At the critical point of B-E condensation,$N_0 \ll N$,$z=1$.
\begin{equation}
\frac{N}{V} = \frac{1}{\lambda^3} \frac{1}{\Gamma(\frac{3}{2})} (\int_0^{\infty} \frac{x^{\frac{1}{2}}dx}{e^x+1} + \int_0^{\infty} \frac{x^{\frac{1}{2}}dx}{e^{x+\beta \epsilon_1}+1})
\end{equation}
Because$\beta \epsilon_1 >> 1$,so 
\begin{equation}
\int_0^{\infty} \frac{x^{\frac{1}{2}}dx}{e^{x+\beta \epsilon_1}+1}= e^{-\beta \epsilon_1}
\int_0^{\infty} e^{-x} x^{\frac{1}{2}} dx = \Gamma(\frac{3}{2}) e^{-\beta \epsilon_1} \ll 1
\end{equation}
So, we think $\Delta T = T_c-T_{c0} \ll 1$.
\begin{equation}
\frac{N}{V} = \frac{1}{\lambda^3} (\xi(\frac{3}{2})+e^{-\beta \epsilon_1})
\approx \frac{1}{\lambda_0^3} (1+\frac{3\Delta T}{2 T_{c0}})(\xi(\frac{3}{2})+e^{\frac{\epsilon_1}{kT_{c0}}})
\end{equation}
And we know
\begin{equation}
\frac{N}{V} = \frac{1}{\lambda_0^3} \xi(\frac{3}{2})
\end{equation}
So,
\begin{equation}
\frac{3\Delta T}{2 T_{c0}} \xi(\frac{3}{2}) + e^{\frac{\epsilon_1}{kT_{c0}}} =0
\end{equation}
\begin{equation}
\frac{T_c}{T_{c0}} = 1 -  \frac{\frac{2}{3}}{\xi(\frac{3}{2})} e^{\frac{\epsilon_1}{kT_{c0}}}
\end{equation}

\subsection*{(b)}
\begin{equation}
\frac{N}{V} = \frac{1}{\lambda^3} \frac{1}{\Gamma(\frac{3}{2})} (\int_0^{\infty} \frac{x^{\frac{1}{2}}dx}{e^x+1} + \int_0^{\infty} \frac{x^{\frac{1}{2}}dx}{e^{x+\beta \epsilon_1}+1}) = \frac{1}{\lambda^3} (g_{3/2}(1) + g_{3/2}(e^{-\beta \epsilon_1}))
\end{equation}
As $\beta \epsilon_1 \ll 1$, we can expand $g_{3/2}(e^{-\beta \epsilon_1})$ and keep the first two terms.
\begin{equation}
\frac{N}{V} = \frac{1}{\lambda^3} (2\xi(\frac{3}{2})+\Gamma(-\frac{1}{2})(\beta \epsilon_1)^{\frac{1}{2}}) = \frac{1}{\lambda_0^3} \xi(\frac{3}{2})
\end{equation}
If $\beta \epsilon_1 = 0$,
\begin{equation}
\lambda' = 2^{\frac{1}{3}} \lambda_0
\end{equation}
\begin{equation}
T'_c = (\frac{1}{2})^{\frac{2}{3}}T_{c0}
\end{equation}
And $T'_{c}-T_c \ll 1$.

\begin{equation}
\frac{N}{V} = \frac{2}{\lambda'^3} (1+\frac{3\Delta T}{2 T'_{c}}) (\xi(\frac{3}{2}) - \sqrt{\pi (\frac{\epsilon_1}{kT'_c}})) = \frac{1}{\lambda_0^3} \xi(\frac{3}{2})
\end{equation}

\begin{equation}
\frac{3\Delta T}{2 T'_{c}} \xi(\frac{3}{2}) - \sqrt{\frac{\pi \epsilon_1}{kT'_c}} =0 
\end{equation}

\begin{equation}
\frac{\Delta T}{T'_{c}} = \frac{\frac{2}{3}}{\xi(\frac{3}{2})}\sqrt{\frac{\pi \epsilon_1}{kT'_c})} =  \frac{2^{\frac{4}{3}}}{3\xi(\frac{3}{2})} \sqrt{\frac{\pi \epsilon_1}{kT_{c0}}}
\end{equation}

\begin{equation}
\frac{T_c}{T_{c0}} = (\frac{1}{2})^{\frac{2}{3}} (1+\frac{2^{\frac{4}{3}}}{3\xi(\frac{3}{2})} \sqrt{\frac{\pi \epsilon_1}{kT_{c0}}})
\end{equation}


\section*{Problem 7.14}
From the definition of p and U:
$$p=\frac{kT}{\lambda ^n}g_{\frac{5}{2}}(z)$$
$$U=\frac{nkTV}{s\lambda ^n}g_{\frac{5}{2}}(z)$$
So we get:
$$p=\frac{sU}{nV}$$
And when $T\rightarrow\infty$,we could use the ideal gas equation.
$$pV=nRT$$
So
$$C_V=\frac{n}{s}Nk,C_p=(\frac{n}{s}+1)Nk$$

\section*{Problem 7.21}
\begin{equation}
\frac{N}{V} = \frac{1}{h^3}\int_0^{\infty} \frac{8\pi \frac{\epsilon^2}{c^3}d\epsilon}{e^{\beta \epsilon}-1} = \frac{8\pi}{\beta^3h^3c^3} \int_0^{\infty} \frac{x^2 dx}{e^{x}-1} =\frac{8\pi}{\beta^3h^3c^3} \Gamma(3) \xi(3)
\end{equation}

\begin{equation}
\frac{E}{V} = \frac{1}{h^3}\int_0^{\infty} \frac{8\pi \frac{\epsilon^3}{c^3}d\epsilon}{e^{\beta \epsilon}-1} = \frac{8\pi}{\beta^4h^3c^3} \int_0^{\infty} \frac{x^3 dx}{e^{x}-1} =\frac{8\pi}{\beta^4h^3c^3} \Gamma(4) \xi(4)
\end{equation}

\begin{equation}
\frac{E}{N} = \frac{3\xi(4)}{\xi(3)} kT \approx 2.7kT
\end{equation}

\section*{Problem 7.24}
We can deduce from 7.3.12 and 7.3.19 and 7.3.23.
$$u=4.16*10^{-14}$$
$$s=2.03*10^{-14}$$
$$n=4.09*10^{8}$$


\section*{Problem 7.31}
The state density for transverse phonon is
\begin{equation}
g(\omega) = \frac{6N}{\omega^3_{DT}} \omega^2
\end{equation}
The state density for longitudinal phonon is
\begin{equation}
g(\omega) = \frac{3N}{\omega^3_{DL}} \omega^2
\end{equation}
And
\begin{equation}
C_V = C_{VT} + C_{VL}
\end{equation}
Compared with equation (7.4.17),it is easy to find verify that
\begin{equation}
C_V = Nk(2D(x_{0,T}) + D(x_{0,L})))
\end{equation}
The nature of the error in equation (7.4.17) comes from the fact that the longitudinal and the transverse modes of the solid should have their own cutoff frequencies, $\omega_{DL}$ and $\omega_{DT}$ rather than a common cutoff at $\omega_{D}$.
\end{document}

\section*{Problem 7.34}
For the n-dimensional Debye system,we can get the function of  state number:
$$g(\omega)\propto
\begin{cases}
0& \text{($\omega>\omega_D$)}\\
\omega^{n-1}& \text{($0<\omega<\omega_D$)}
\end{cases}$$
Then we consider the energy of the Debye system:
$$U_{ph}\propto \int_0^{\omega_D}\frac{\omega^n}{exp(\beta\hbar\omega)-1}d\omega\propto T^{n+1}$$
We can know the specific heat from thermodynamics:
$$C_V=(\frac{\partial U}{\partial T})_V\propto T^n$$





\end{document}

