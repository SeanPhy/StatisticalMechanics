\documentclass{article}

\usepackage[left=1.5cm, right=1.5cm, top=3cm, bottom = 3cm]{geometry}

\usepackage{amsmath}
\usepackage{amsfonts}
\usepackage{amssymb}
\usepackage{graphicx}
\usepackage{float}
\usepackage{wrapfig}
\usepackage{latexsym}
\usepackage{hyperref}
\usepackage{feynmf}
\usepackage{exscale}
\usepackage{relsize}
\usepackage{bm}%bold math, for vector
\linespread{1.1}

%%%%%%%
%第⑦章习题安排:
%%%%%%%
%宋志坚 1,2,3
%宋盛雨央 11,21,31
%陈博文 12,22,32
%解放 13,23,33
%辜晨曦 4,14,24,34
%鲍亦澄 5,15,25,35
%蒋文韬 6,16,26,36
%李嘉琛 7,17,27,37
%颜公望 8,18,28,38
%张传坤 9,19,29
%王志凌 10,20,30

\author{SM-at-THU}
\title{\bf{Solutions to Pathria's Statistical Mechanics}\\Chapter 7}

\begin{document}
\maketitle
\section*{Problem 7.1}

\section*{Problem 7.4}
We can deduce from 7.1.30 and 7.1.26a,when p is a const:
$$g_{\frac{5}{2}}\propto T^{-\frac{5}{2}}$$
So we have:
$$(\frac{\partial}{\partial T}{g_{\frac{5}{2}}(z)})_p=-\frac{5}{2T}  g_{\frac{5}{2}}(z)$$
According to the D.10 from appendix D:
$$z\frac{\partial}{\partial z}g_{\frac{5}{2}}(z)=g_{\frac{3}{2}}(z)$$
Combine the two equation:
$$\frac{1}{z}(\frac{\partial z}{\partial T})_p=-\frac{5}{2T}\frac{g_{\frac{5}{2}}(z)}{g_{\frac{3}{2}}(z)}$$
If you compare with 7.1.36 you can get:
$$\gamma=\frac{5}{3}\frac{g_{\frac{5}{2}}(z)g_{\frac{1}{2}}(z)}{g_{\frac{3}{2}}(z)^2}$$


\section*{Problem 7.14}
From the definition of p and U:
$$p=\frac{kT}{\lambda ^n}g_{\frac{5}{2}}(z)$$
$$U=\frac{nkTV}{s\lambda ^n}g_{\frac{5}{2}}(z)$$
So we get:
$$p=\frac{sU}{nV}$$
And when $T\rightarrow\infty$,we could use the ideal gas equation.
$$pV=nRT$$
So
$$C_V=\frac{n}{s}Nk,C_p=(\frac{n}{s}+1)Nk$$


\section*{Problem 7.24}
We can deduce from 7.3.12 and 7.3.19 and 7.3.23.
$$u=4.16*10^{-14}$$
$$s=2.03*10^{-14}$$
$$n=4.09*10^{8}$$


\section*{Problem 7.34}
For the n-dimensional Debye system,we can get the function of  state number:
$$g(\omega)\propto
\begin{cases}
0& \text{($\omega>\omega_D$)}\\
\omega^{n-1}& \text{($0<\omega<\omega_D$)}
\end{cases}$$
Then we consider the energy of the Debye system:
$$U_{ph}\propto \int_0^{\omega_D}\frac{\omega^n}{exp(\beta\hbar\omega)-1}d\omega\propto T^{n+1}$$
We can know the specific heat from thermodynamics:
$$C_V=(\frac{\partial U}{\partial T})_V\propto T^n$$





\end{document}

