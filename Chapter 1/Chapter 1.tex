\documentclass{article}

\usepackage[left=1.5cm, right=1.5cm, top=3cm, bottom = 3cm]{geometry}

\usepackage{amsmath}
\usepackage{amsfonts}
\usepackage{amssymb}
\usepackage{graphicx}
\usepackage{float}
\usepackage{indentfirst}
\usepackage{wrapfig}
\usepackage{latexsym}
\usepackage{hyperref}
\usepackage{feynmf}
\linespread{1.1}

\author{SM-at-THU}
\title{\bf{Solutions to Pathria's Statistical Mechanics}\\Chapter 1}

\begin{document}
\maketitle
\section*{Problem 1.1}

\section*{Problem 1.2}
Utilizing the additive characteristic of $S=f(\Omega)$ and get
\begin{align}
&S=S_1+S_2=f(\Omega_1)+f(\Omega_2)\\
&(\frac{dS}{d\Omega_1})_{\Omega_2}=f'(\Omega_1)\\
&(\frac{dS}{d\Omega_2})_{\Omega_1}=f'(\Omega_2)\\
\end{align}

Inspect a small pertubation near the equilibrium state using the fact that $S=f(\Omega)=f(\Omega_1\Omega_2)$
\begin{equation}
(\frac{dS}{d\Omega_1})_{\Omega_2}=\lim_{\Delta\rightarrow 0}{\frac{f((\Omega_1+\Delta)\Omega_2)-f(\Omega_1\Omega_2)}{\Delta}}
\end{equation}
Assume that $\delta=\Delta\Omega_2$
\begin{equation}
(\frac{dS}{d\Omega_1})_{\Omega_2}=\lim_{\Delta\rightarrow 0}{\Omega_2\frac{f(\Omega_1\Omega_2+\Delta\Omega_2)-f(\Omega_1\Omega_2)}{\Delta\Omega_2}}=\lim_{\delta\rightarrow 0}{\Omega_2\frac{f(\Omega+\delta)-f(\Omega)}{\delta}}=\Omega_2f'(\Omega)
\end{equation}
Apply to $(\frac{dS}{d\Omega_2})_{\Omega_1}$, we can get similar result.
\begin{equation}
(\frac{dS}{d\Omega_2})_{\Omega_1}=\Omega_1f'(\Omega)
\end{equation}
Finally,
\begin{align}
f'(\Omega_1)&=\Omega_2f'(\Omega)=\frac{\Omega_2}{\Omega_1}f'(\Omega_2)\\
\Omega_1f'(\Omega_1)&=\Omega_2f'(\Omega_2)\\
\end{align}
It is obvious that this equation holds for all $\Omega$. Set the value of the equation constant k. 
\begin{align}
\Omega\frac{df(\Omega)}{d\Omega}&=k\\
f(\Omega)&=kln\Omega+C\\
\end{align}
Using a special value $\Omega=1$
\begin{align}
f(\Omega*1)&=f(\Omega)+f(1)\\
C=&f(1)=0\\
\end{align}
And get the result 
\begin{equation}
S=f(\Omega)=kln\Omega
\end{equation}
\section*{Problem 1.3}



\section*{Problem 1.4}
Suppose $N$ is the number of particles, $v_{0}$ is the volume occupied by one particle and therefore the total number of microstates $\Omega$ is
\begin{equation}
\Omega = \frac{1}{N!}(\frac{V}{v_{0}}) \dots (\frac{V}{v_{0}}-N+1)
\end{equation}
Following $(1.4.2)$, we have
\begin{eqnarray}
\frac{P}{T} &=& k \left(\frac{\partial \ln \Omega}{\partial V}\right)_{N,E} \\
&=& k \frac{\partial \Omega}{\Omega \partial V} \\
&=& k \frac{N}{V} \left(1+ \frac{(N-1)v_{0}}{2V} + \dots \right)
\end{eqnarray}
Considering only the first two terms, it corresponds to $P(V-b) = NkT$ with $b=N v_{0}/2$. \\
Notes: I don't know why the problem says $b=4N v_{0}$ since this gas is hard sphere gas. Anyone has an idea?


\section*{Problem 1.5}
Using equation $(A.11)$, and setting $K = \pi \sqrt{\varepsilon}/L$, it is straight forward to achieve
\begin{equation}
\Sigma_{1}(\varepsilon) = \frac{\pi}{6}\varepsilon^{3/2} \pm \frac{3 \pi}{8} \varepsilon
\end{equation}
where the first term is the volume term ($V=L^{3}$) and the next one is the surface correction ($S=6L^{2}$).



\section*{Problem 1.6}
Use the formula for ideal gas $PV=NkT$.
\begin{equation}
N k \times 300= 10^{5} \times \frac{\pi}{10}
\end{equation}
Thus $\Delta T = 10^{4}/Nk \sim 955 K$.


\end{document}