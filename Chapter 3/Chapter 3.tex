\documentclass{article}

\usepackage[left=1.5cm, right=1.5cm, top=3cm, bottom = 3cm]{geometry}

\usepackage{amsmath}
\usepackage{amsfonts}
\usepackage{amssymb}
\usepackage{graphicx}
\usepackage{float}
\usepackage{wrapfig}
\usepackage{latexsym}
\usepackage{hyperref}
\usepackage{feynmf}
\linespread{1.1}

%%%%%%%
%第三章习题安排:
%%%%%%%
%宋志坚 1,2,3,4
%宋盛雨央 10,20,30,40
%陈博文 11,21,31,41
%解放 12,22,32,42
%辜晨曦 13,23,33,43
%鲍亦澄 14,24,34,44
%蒋文韬 5,15,25,35
%李嘉琛 6,16,26,36
%颜公望 7,17,27,37
%张传坤 8,18,28,38
%王志凌 9,19,29,39

\author{SM-at-THU}
\title{\bf{Solutions to Pathria's Statistical Mechanics}\\Chapter 3}

\begin{document}
\maketitle

\section*{Problem 3.1}

\section*{Problem 3.2}

\section*{Problem 3.3}

\section*{Problem 3.4}

\section*{Problem 3.5}

\section*{Problem 3.6}

\section*{Problem 3.7}

\section*{Problem 3.11}
Suppose $pV^{n} = C$, so the work done is
\begin{equation}
\Delta W = \int^{V_{2}}_{V_{1}} \frac{C}{V^{n}} dV = \frac{C}{n-1} (V^{1-n}_{2} - V^{1-n}_{1})
\end{equation}
The energy difference is given by
\begin{equation}
\Delta U = p_{2}V_{2} - p_{1}V_{1} = C (V^{1-n}_{2} - V^{1-n}_{1})
\end{equation}
Therefore, the heat absorbed is
\begin{equation}
\Delta Q =  C\frac{n-2}{n-1} (V^{1-n}_{2} - V^{1-n}_{1})
\end{equation}

\section*{Problem 3.21}
\noindent (a) Classically, the harmonic equation of motion leads to $x = A \sin \omega t$. As a result, the kinetic energy and potential energy will be $m \omega^{2} A^{2} \cos^{2} \omega t /2$ and $m \omega^{2} A^{2} \sin^{2} \omega t /2$ respectively. Average them it's easy to see that $\bar{K} = \bar{U} =m \omega^{2} A^{2}/4$.\\
Quantum-mechanically, $\psi = \sum_{n} c_{n} \psi_{n}$ where $\psi_{n}$ is the \emph{n}-th Hermitian polynomial. Using the recursive relations, we have
\begin{align}
&\bar{K} = -\frac{\hbar^{2}}{2m} \sum_{n} |c_{n}|^{2} \int \psi^{*} \frac{d^{2}}{dx^{2}} \psi dx = \sum_{n} |c_{n}|^{2} \frac{\hbar \omega (2n+1)}{4} = \frac{1}{2} \sum_{n} |c_{n}|^{2} E_{n}\\
&\bar{U} = \frac{m \omega^{2}}{2} \sum_{n} |c_{n}|^{2} \int \psi^{*} x^{2} \psi dx= \sum_{n} |c_{n}|^{2} \frac{\hbar \omega (2n+1)}{4} = \frac{1}{2} \sum_{n} |c_{n}|^{2} E_{n}
\end{align}
\noindent (b) In Bohr-sommerfeld model, a quantized orbits are hypothesized, namely $m_{e}vr = n \hbar$. In the \emph{n}-th orbit, the total energy is $E_{n} = - Z^{2}k^{2}e^{4}m_{e}/2 \hbar^{2} n^{2}$. The radius of which is $r_{n} = n^{2} \hbar^{2} / Zke^{2}m_{e}$. By a naive calculation $\bar{U} = - Z^{2}k^{2}e^{4}m_{e}/ \hbar^{2} n^{2}$ and $\bar{T} = Z^{2}k^{2}e^{4}m_{e}/2 \hbar^{2} n^{2}$.\\
In the Schroedinger hydrogen atom, $\psi_{nlm} = R_{nl}(r) Y_{lm}(\theta, \phi)$. The kinetic energy is given by
\begin{eqnarray}
\bar{T} &=& \frac{\hbar^{2}}{2m} \int \psi^{*}_{nlm} (\frac{d^{2}}{dr^{2}} + \frac{2}{r} \frac{d}{dr} - \frac{l(l+1)}{r^{2}})\psi_{nlm} r^{2} \sin \theta dr d\theta d\phi \notag \\
&=& \frac{\hbar^{2}}{2m} \int R_{nl}(r) (\frac{1}{n^{2}a^{2}}) R_{nl}(r) r^{2}dr \notag \\
&=& \frac{e^{2}}{2a n^{2}}
\end{eqnarray}
so $\bar{U} = - e^{2}/a n^{2}$. \emph{a} is the Bohr radius.\\
\noindent (c) This is also a central force case. The results are quite identical to (b).

\section*{Problem 3.31}
``Partition function'' for single particle is
\begin{equation}
Q_{1} = 1 + e^{-\varepsilon/kT}.
\end{equation}
So a list of quatities can be obtained:
\begin{align}
&Q_{N} = (1 + e^{-\varepsilon/kT})^{N} \\
&A = - NkT \ln (1 + e^{-\varepsilon/kT}) \\
&\mu = - kT \ln (1 + e^{-\varepsilon/kT}) \\
&p=0 \\
&S = Nk \ln (1 + e^{-\varepsilon/kT}) + \frac{N \varepsilon}{T} \frac{e^{-\varepsilon/kT}}{1+e^{-\varepsilon/kT}} \\
&U =  N \varepsilon \frac{e^{-\varepsilon/kT}}{1+e^{-\varepsilon/kT}} \\
&C_{p} = C_{V} =  \frac{N \varepsilon^{2} e^{-\varepsilon/kT}}{kT^{2} (1+e^{-\varepsilon/kT})^{2}}
\end{align}
This specific heat is sometimes referred to \emph{Schottky anomaly}. 


\section*{Problem 3.41}
The equilibrium temperature will be positive, since the energy of the whole system is not bounded from above. This case is a bit like the spin and lattice case. For the subsystem of spins, its energy is bounded from above, so it is possible to attain a negative temperature. While the subsystem of lattice, i.e. ideal gas in this problem, only has positive temperature. The whole system doesn't have a energy limit, so the temperature will only be positive. And energy may flow from the spin subsystem to the ideal gas.

\end{document}