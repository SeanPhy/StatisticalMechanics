\documentclass{article}

\usepackage[left=1.5cm, right=1.5cm, top=3cm, bottom = 3cm]{geometry}

\usepackage{amsmath}
\usepackage{amsfonts}
\usepackage{amssymb}
\usepackage{graphicx}
\usepackage{float}
\usepackage{wrapfig}
\usepackage{latexsym}
\usepackage{hyperref}
\usepackage{feynmf}
\linespread{1.1}

%%%%%%%
%第三章习题安排:
%%%%%%%
%宋志坚 1,2,3,4
%宋盛雨央 10,20,30,40
%陈博文 11,21,31,41
%解放 12,22,32,42
%辜晨曦 13,23,33,43
%鲍亦澄 14,24,34,44
%蒋文韬 5,15,25,35
%李嘉琛 6,16,26,36
%颜公望 7,17,27,37
%张传坤 8,18,28,38
%王志凌 9,19,29,39

\author{SM-at-THU}
\title{\bf{Solutions to Pathria's Statistical Mechanics}\\Chapter 3}

\begin{document}
\maketitle

\section*{Problem 3.1}

\section*{Problem 3.2}

\section*{Problem 3.3}

\section*{Problem 3.4}

\section*{Problem 3.5}

	Since the Helmholtz free energy $A(N,V,T)$ has the property:
	\begin{equation*}
		A(\lambda N,\lambda V,T) = \lambda A(N,V,T)
	\end{equation*}
	Differentiate with respect to $\lambda$ and substitute $\lambda=1$ immediately yields
	\begin{equation*}
		N\left( \frac{ \partial A }{\partial N} \right)_{V,T}+V \left( \frac{ \partial A }{\partial V} \right)_{N,T} = A
	\end{equation*}

\section*{Problem 3.6}

\section*{Problem 3.7}

\end{document}