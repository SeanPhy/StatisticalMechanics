\documentclass{article}

\usepackage[left=1.5cm, right=1.5cm, top=3cm, bottom = 3cm]{geometry}

\usepackage{amsmath}
\usepackage{amsfonts}
\usepackage{amssymb}
\usepackage{graphicx}
\usepackage{float}
\usepackage{wrapfig}
\usepackage{latexsym}
\usepackage{hyperref}
\usepackage{feynmf}
\usepackage{exscale}
\usepackage{relsize}
\linespread{1.1}

%%%%%%%
%第五章习题安排:
%%%%%%%
%宋志坚 1
%宋盛雨央 2
%陈博文 3
%解放 4
%辜晨曦 5
%鲍亦澄 6
%蒋文韬 
%李嘉琛 
%颜公望 
%张传坤 7
%王志凌 8

\author{SM-at-THU}
\title{\bf{Solutions to Pathria's Statistical Mechanics}\\Chapter 5}

\begin{document}
\maketitle
\section*{Problem 5.1}

\section{Problem 5.5} % (fold)
\label{sec:problem_5_5}
	
	The partition function of the noninteracting,indistinguishable particles system is:
	$$Q_N(V,T)=\frac{1}{N!\lambda^{3N}}\sum_P \delta_P[f(Pr_1-r_1)\cdots f(Pr_N-r_N)]$$
	where $f(r)=e^{-\frac{\pi r^2}{\lambda^2}}$
	The first approximation is:
	$$\sum_P=1\pm\sum_{i<j}f_{ij}f_{ji}$$
	So the partition function in first approximation is:
	$$Q_N(V,T)=\frac{1}{N!\lambda^{3N}}\int(1\pm\sum_{i<j}e^{-\frac{\pi r_{ij}^2}{\lambda^2}})d^{3N}r$$
% section problem_5_5 (end)

\end{document}